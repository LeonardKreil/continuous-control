\section{Introduction}

\lettrine[nindent=0em,lines=3]{R} einforcement Learning (RL) has demonstrated remarkable successes in various domains, from achieving superhuman performance in games like Go and chess \cite{doi:10.1126/science.aar6404} to optimizing complex industrial processes \cite{NEURIPS2018_059fdcd9}. While discrete action spaces have been well-studied in RL literature, many real-world applications require agents to operate in continuous action spaces which is a significantly more challenging domain known as continuous control \cite{lillicrap2019continuouscontroldeepreinforcement}.  In continuous control, the agent must learn to select actions from an infinite set of possibilities, rather than choosing from a finite set of discrete options.\\

\noindent The field of continuous control has gained substantial attention due to its direct applicability to robotics, autonomous vehicles, and other physical systems. These domains require precise, nuanced control signals rather than discrete choices. For example, a walking robot needs to apply specific torque values to each joint rather than simply selecting "move forward" or "turn left". This continuous nature introduces several fundamental challenges to reinforcement learning algorithms.\\

\noindent Continuous control is particularly challenging for several reasons. First, exploration in continuous spaces is inherently more difficult than in discrete spaces, as the agent must effectively search through an infinite action space \cite{duan2016benchmarkingdeepreinforcementlearning}. Second, function approximation becomes necessary to represent policies and value functions over continuous domains, introducing additional complexity and stability concerns. Third, continuous control tasks often involve complex dynamics and long time horizons, requiring algorithms to learn long-term dependencies \cite{lillicrap2019continuouscontroldeepreinforcement}.\\

\noindent The Bipedal Walker environment from OpenAI Gym represents an exemplary continuous control challenge, embodying many of the difficulties found in complex physical systems. In this environment, an agent must learn to control a two-legged robot to walk forward without falling, applying continuous torque values to each joint. The task combines the challenges of balance and coordination, making it an excellent benchmark for comparing continuous control algorithms.\\

\noindent This paper aims to provide a comprehensive comparison of three prominent continuous control algorithms, the Deep Deterministic Policy Gradient (DDPG) \cite{lillicrap2019continuouscontroldeepreinforcement}, the Soft Actor-Critic (SAC) \cite{haarnoja2018softactorcriticoffpolicymaximum}, and the Proximal Policy Optimization (PPO) \cite{schulman2017proximalpolicyoptimizationalgorithms} in the context of the Bipedal Walker environment. By analyzing their performance, convergence properties, and stability characteristics, we seek to identify which approaches are most effective for this challenging locomotion task and potentially for similar continuous control problems.\\
